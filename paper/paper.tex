\documentclass[acmtoplas]{acmtrans2m}

\usepackage{amsmath}
\usepackage{amssymb}
\usepackage{wasysym}
\usepackage{booktabs}
\usepackage{proof}
\usepackage{twelf}
\usepackage[usenames,dvipsnames,svgnames,table]{xcolor}
\usepackage{code}
\usepackage{scalerel}

\newtheorem{theorem}{Theorem}[section]
\newtheorem{conjecture}[theorem]{Conjecture}
\newtheorem{corollary}[theorem]{Corollary}
\newtheorem{proposition}[theorem]{Proposition}
\newtheorem{lemma}[theorem]{Lemma}
\newtheorem{case}{Case}
\newdef{definition}[theorem]{Definition}
\newdef{remark}[theorem]{Remark}
\def\permission{}

\def\InputModeColorName{MidnightBlue}
\def\OutputModeColorName{Maroon}
\newcommand\InputMode[1]{{\color{\InputModeColorName}{#1}}}
\newcommand\OutputMode[1]{{\color{\OutputModeColorName}{#1}}}

\newcommand\naturals{\mathbb{N}}
\newcommand\type{\mathbf{type}}
\newcommand\tyvoid{\mathsf{void}}
\newcommand\tyunit{\mathsf{unit}}
\newcommand\ax{\mathsf{ax}}
\newcommand\ap{\mathsf{ap}}
\newcommand\sortval{\mathsf{val}}
\newcommand\sortexp{\mathsf{exp}}
\newcommand\sortctx{\mathsf{ctx}}
\newcommand\sorttele{\mathsf{tele}}
\newcommand\sortoexp[1]{[#1]\sortexp}
\newcommand\sortj{\mathsf{judg}}
\newcommand\sortoj[1]{[#1]\sortj}
\newcommand\eval[2]{\InputMode{#1}\mathbin{\Rightarrow}\OutputMode{#2}}
\newcommand\thunk[1]{{\color{Gray}\uparrow}#1}
\newcommand\evalrule[1]{\Rightarrow_{#1}}
\newcommand\nobind[1]{{\color{Gray}!}#1}
\newcommand\bind[2]{{\color{Gray}@}[#1]#2}
\newcommand\tacitbind[1]{{\color{Gray}@}#1}

\newcommand\isset[1]{\InputMode{#1}\;\mathit{set}}
\newcommand\eqset[2]{\InputMode{#1}=\InputMode{#2}\;\mathit{set}}
\newcommand\mem[2]{\InputMode{#1}\in\InputMode{#2}}
\newcommand\eqmem[3]{\InputMode{#1}=\InputMode{#2}\in\InputMode{#3}}

\newcommand\canset[1]{\InputMode{#1}\;\mathit{set}_\mathsf{v}}
\newcommand\eqcanset[2]{\InputMode{#1}=\InputMode{#2}\;\mathit{set}_\mathsf{v}}
\newcommand\canmem[2]{\InputMode{#1}\in_\mathsf{v}\InputMode{#2}}
\newcommand\eqcanmem[3]{\InputMode{#1}=\InputMode{#2}\in_\mathsf{v}\InputMode{#3}}
\newcommand\lfpi[2]{{\color{Gray}\forall\{#1\}}.\;#2}

\newcommand\weaken[3]{\mathsf{wk}_{\InputMode{#1}}\,\InputMode{#2}\leadsto\OutputMode{#3}}
\newcommand\weakenrulenil{\mathsf{wk}_\mathsf{nil}}
\newcommand\weakenrulesnoc{\mathsf{wk}_\mathsf{snoc}}
\newcommand\qisset[1]{#1\;`\mathit{set}}
\newcommand\qeqset[2]{#1=#2\;`\mathit{set}}
\newcommand\qmem[2]{#1\;`\!\!\in#2}
\newcommand\qeqmem[3]{#1=#2\;`\!\!\in#3}

\DeclareMathOperator{\typi}{\mathsf{pi}}
\DeclareMathOperator{\tysg}{\mathsf{sg}}
\DeclareMathOperator{\lam}{\mathsf{\lambda}}
\DeclareMathOperator{\pair}{\mathsf{pair}}
\DeclareMathOperator{\fst}{\mathsf{fst}}
\DeclareMathOperator{\snd}{\mathsf{snd}}

\newcommand\sploot[2]{\InputMode{#1}\triangleright\OutputMode{#2}}
\newcommand\refract[2]{\InputMode{#1}\ltimes\OutputMode{#2}}

\newcommand\sequent[2]{\InputMode{#1}\gg\InputMode{#2}}
\newcommand\nece[1]{\gg\InputMode{#1}}

\newcommand\sequentruleset{\gg_\mathsf{set}}
\newcommand\sequentruleeqset{\gg_{\mathsf{set}_=}}
\newcommand\sequentrulemem{\gg_\in}
\newcommand\sequentruleeqmem{\gg_{\in_=}}
\newcommand\dname[1]{\mathcal{#1}}
\newcommand\dlabel[2]{\begin{array}{c}#1\\#2\end{array}}

%% Crary Stuff
\DeclareMathOperator\isvarop{\mathsf{isvar}}
\DeclareMathOperator\boundedop{\mathsf{bounded}}
\DeclareMathOperator\orderedop{\mathsf{ordered}}
\DeclareMathOperator\precedesop{\mathsf{precedes}}

\newcommand\isvar[2]{\isvarop(\InputMode{#1};\OutputMode{#2})}
\newcommand\bounded[2]{\boundedop(\InputMode{#1};\InputMode{#2})}
\newcommand\ordered[1]{\orderedop(\InputMode{#1})}
\newcommand\precedes[2]{\precedesop(\InputMode{#1};\InputMode{#2})}
\newcommand\boundedrule[1]{\mathsf{bounded}_{#1}}
\newcommand\orderedrule[1]{\mathsf{ordered}_{#1}}
\newcommand\lessthan[2]{\InputMode{#1}<\InputMode{#2}}

\title{A Higher-Order Approach to Explicit Contexts in LF}

\author{JONATHAN M.\ STERLING}

\begin{abstract}

  There are two standard techniques for representing dependently typed calculi
  in the LF. The first is to re-use the LF contexts for the object contexts,
  which can lead to difficulties when the meaning of hypothetico-general
  judgement in the object language is stronger than in the LF; this is, for
  instance, the case in MLTT~1979, where non-trivial functionality obligations
  are incurred in the sequent judgement.

  Another technique advanced by Crary is to represent contexts explicitly, and
  then define a sequent judgement over them; this can be used to resolve the
  problem described above (and several others), but it comes at the cost of
  verbosity, and introduces difficulties when used to encode \emph{telescopes}
  in the LF, which are contexts where where each successive term has the
  previous bound variables free in it.  In this paper, I demonstrate a
  higher-order encoding of telescopes which can be used to faithfully encode
  the functional sequent judgement for MLTT~1979.

\end{abstract}

\category{}{}{}
\terms{Logical Frameworks}
\keywords{Contexts, telescopes}

\begin{document}

\maketitle

\section{Introduction}

To motivate this work, let us consider an encoding of a fragment of
Martin-L\"of's extensional, polymorphic type
theory\cite{Martin-Lof-1979} in the logical framework. The syntax
contains two sorts, $\sortval$ (for canonical forms) and $\sortexp$
(for non-canonical forms), and is defined in the following LF
signature:\footnote{LF signatures are presented in the style of
inference rules for clarity. Object judgement forms have a \emph{mode}
which expresses which arguments represent inputs, and which represent
outputs; in our presentation, we color the inputs
\emph{\InputMode\InputModeColorName}, and the outputs
\emph{\OutputMode\OutputModeColorName}. Proper assignment of modes may
be used to construe a judgement as having algorithmic content.}

\begin{gather*}
  \infer{\sortval:\type}{}
  \quad
  \infer{\sortexp:\type}{}\\[6pt]
  %
  \infer{\tyvoid:\sortval}{}
  \quad
  \infer{\tyunit:\sortval}{}
  \quad
  \infer{
    \typi(A;[x]B(x)):\sortval
  }{
    A:\sortexp &
    B:\sortexp\to\sortexp
  }
  \quad
  \infer{
    \tysg(A;[x]B(x)):\sortval
  }{
    A:\sortexp &
    B:\sortexp\to\sortexp
  }\\[6pt]
  %
  \infer{\ax:\sortval}{}
  \quad
  \infer{
    \lam[x]E(x):\sortval
  }{
    E:\sortexp\to\sortexp
  }
  \quad
  \infer{
    \pair(M;N):\sortval
  }{
    M:\sortexp &
    N:\sortexp
  } \\[6pt]
  %
  \infer{
    \thunk{M}:\sortexp
  }{
    M:\sortval
  }
  \quad
  \infer{
    \ap(M;N):\sortexp
  }{
    M:\sortexp &
    N:\sortexp
  }
  \quad
  \infer{
    \fst(M):\sortexp
  }{
    M:\sortexp
  }
  \quad
  \infer{
    \snd(M):\sortexp
  }{
    M:\sortexp
  }
\end{gather*}


Then, the language is accorded a simple big-step operational semantics via the
\framebox{$\eval{M}{N}$} judgement:

\begin{gather*}
  \infer{
    \eval{M}{N}:\type
  }{
    M:\sortexp &
    N:\sortval
  }\\[6pt]
  %
  \infer[\evalrule\uparrow]{
    \eval{\thunk{M}}{M}
  }{
  }
  \quad
  \infer[\evalrule\ap]{
    \eval{\ap(L;M)}{N}
  }{
    \eval{L}{\lam[x]E(x)} &
    \eval{E(M)}{N}
  }\\[6pt]
  %
  \infer[\evalrule\fst]{
    \eval{\fst(M)}{M_1'}
  }{
    \eval{M}{\pair(M_1;M_2)} &
    \eval{M_1}{M_1'}
  }
  \quad
  \infer[\evalrule\snd]{
    \eval{\snd(M)}{M_2'}
  }{
    \eval{M}{\pair(M_1;M_2)} &
    \eval{M_2}{M_2'}
  }
\end{gather*}

\section{The Na\"ive Higher-Order Approach}

Informally, there are four basic categorical judgements in MLTT~1979:\\

\begin{tabular}{ll}
  \toprule
  Form & Pronunciation\\
  \midrule
  $\isset{A}$ & $A$ is a set\\
  $\eqset{A}{B}$ & $A$ and $B$ are equal sets\\
  $\mem{M}{A}$ & $M$ is a member of $A$\\
  $\eqmem{M}{N}{A}$ & $M$ and $N$ are equal members of $A$\\
  \bottomrule
\end{tabular}\\

\medskip\noindent The subjects of these judgements are terms of sort
$\sortexp$. However, implicit in Martin-L\"of's presentation are four prior
judgements which operate only on canonical forms (i.e. terms of sort
$\sortval$). We will therefore introduce the following LF signature to account
for this:

\begin{gather*}
  \infer{
    \canset{A}:\type
  }{
    A:\sortval
  }\quad
  \infer{
    \eqcanset{A}{B}:\type
  }{
    A:\sortval &
    B:\sortval
  }\quad
  \infer{
    \canmem{M}{A}:\type
  }{
    M:\sortval &
    A:\sortval
  }\quad
  \infer{
    \eqcanmem{M}{N}{A}:\type
  }{
    M:\sortval &
    N:\sortval &
    A:\sortval
  }\\[6pt]
  \infer{
    \isset{A}:\type
  }{
    A:\sortval
  }\quad
  \infer{
    \eqset{A}{B}:\type
  }{
    A:\sortexp &
    B:\sortexp
  }\quad
  \infer{
    \mem{M}{A}:\type
  }{
    M:\sortexp &
    A:\sortexp
  }\quad
  \infer{
    \eqmem{M}{N}{A}:\type
  }{
    M:\sortexp &
    N:\sortexp &
    A:\sortexp
  }
\end{gather*}

Then, we will give the meanings of the judgements which deal with non-canonical
forms first, anticipating the explanations of the canonical forms judgements.

\begin{gather*}
  \infer{
    \isset{A}
  }{
    \eval{A}{A'} &
    \canset{A'}
  }\quad
  \infer{
    \eqset{A}{B}
  }{
    \begin{array}{c}
      \eval{A}{A'} \\
      \eval{B}{B'}
    \end{array} &
    \eqcanset{A'}{B'}
  }\\[6pt]
  \infer{
    \mem{M}{A}
  }{
    \begin{array}{c}
      \eval{M}{M'}\\
      \eval{A}{A'}
    \end{array} &
    \canmem{M'}{A'}
  }\quad
  \infer{
    \eqmem{M}{N}{A}
  }{
    \begin{array}{c}
      \eval{M}{M'} \\
      \eval{N}{N'} \\
      \eval{A}{A'}
    \end{array} &
    \eqcanmem{M'}{N'}{A'}
  }
\end{gather*}

Following Martin-L\"of, we give an extensional equality for all types (i.e.\
two types are equal when they have the same membership and equivalence
relations); we could choose an intensional/structural equality for types, but
this question is orthogonal to the technique being demonstrated.

\begin{gather*}
  \infer{
    \eqcanset{A}{B}
  }{
    \begin{array}{l}
      \lfpi{M}{\canmem{M}{A}\to\canmem{M}{B}}\\
      \lfpi{M}{\canmem{M}{B}\to\canmem{M}{A}}\\
      \lfpi{M,N}{\eqcanmem{M}{N}{A}\to\eqcanmem{M}{N}{B}}\\
      \lfpi{M,N}{\eqcanmem{M}{N}{B}\to\eqcanmem{M}{N}{A}}
    \end{array}
  }
\end{gather*}

Then, all we have to do is define the remaining canonical forms judgements with
respect to our syntax. For the base types, this is trivial:

\begin{gather*}
  \infer{
    \canset\tyvoid
  }{
  }\quad
  \infer{
    \canset\tyunit
  }{
  }\quad
  \infer{
    \canmem\ax\tyunit
  }{
  }\quad
  \infer{
    \eqcanmem\ax\ax\tyunit
  }{
  }
\end{gather*}

But for terms which involve binding, the rules will become more complicated.
Informally, Martin-L\"of gives the following formation rule for $\mathsf\Pi$
using hypothetico-general judgement:
\[
  \infer[*]{
    \canset{(\mathsf{\Pi}x\in A)B}
  }{
    \isset{A} &
    \isset{B(x)}\;(\mem{x}{A})
  }
\]
So we might be tempted to do something similar in the LF, as follows:
\[
  \infer[*']{
    \canset{\typi(A;[x]B(x))}
  }{
    \isset{A} &
    \lfpi{x}\mem{x}{A}\to\isset{B(x)}
  }
\]

This encoding, however, is not adequate, since the meaning of
hypothetico-general judgement in Martin-L\"of's informal metalangauge is not
captured by the above attempt. Recall that hypothetico-general judgement in
type theory implicitly requires \emph{functionality} of type and term families;
therefore, in order for our definition to be correct, we would have to adjust
the rule to include functionality:
\[
  \infer{
    \canset{\typi(A;[x]B(x))}
  }{
    \isset{A} &
    \begin{array}{l}
      \lfpi{x}\mem{x}{A}\to\isset{B(x)}\\
      \lfpi{x,y}\eqmem{x}{y}{A}\to\eqset{B(x)}{B(y)}\\
    \end{array}
  }
\]

We have at least got closer, but even this is not really satisfactory; we were
unable to factor out the functionality constraints, and must be careful to
place them in all the right places in each and every rule. Contrast this with
Martin-L\"of's definition of hypothetico-general judgement, which includes
functionality from the start, allowing the other judgements to use this ``off
the shelf''.

\section{Explicit Contexts \`a la Crary}
In order to factor out functionality constraints, it will be necessary to
consider the use of contexts in the encoding, and then recast the judgements
above as being sequents, i.e.\ judgements with respect to a context. Crary's
first-order encoding of explicit contexts in the LF is essentially the
following:

\begin{gather*}
  \infer{\sortctx:\type}{}
  \quad
  \infer{\cdot:\sortctx}{}
  \quad
  \infer{
    (\Gamma,X:A):\sortctx
  }{
    \Gamma:\sortctx &
    X:\sortexp &
    A:\sortexp
  }\\[6pt]
  \infer{
    \isvar{X}{I} : \type
  }{
    X:\sortexp &
    I:\naturals
  }
  \quad
  \infer{
    \precedes{X}{Y}: \type
  }{
    X:\sortexp &
    Y:\sortexp
  }\quad
  \infer{
    \precedes{X}{Y}
  }{
    \isvar{X}{I} &
    \isvar{Y}{J} &
    \lessthan{I}{J}
  }\\[6pt]
  \infer{
    \bounded{\Gamma}{X}:\type
  }{
    \Gamma:\sortctx &
    X:\sortexp
  }
  \quad
  \infer{
    \bounded{\cdot}{X}
  }{
    \isvar{X}{I}
  }
  \quad
  \infer{
    \bounded{\Gamma,Y:A}{X}
  }{
    \precedes{Y}{X} &
    \bounded{G}{Y}
  }
  \\[6pt]
  \infer{
    \ordered\Gamma:\type
  }{
    \Gamma:\sortctx
  }
  \quad
  \infer{\ordered\cdot}{}
  \quad
  \infer{
    \ordered{\Gamma,X:A}
  }{
    \bounded{\Gamma}{X}
  }
  \\[6pt]
\end{gather*}

Note that the well-formedness constraints above do not in practice protrude
into the definitions of judgements which \emph{use} explicit contexts; they are
only used in metatheorems. This encoding is very practical for certain calculi,
but it is concerning in our case for a few reasons:

\begin{enumerate}

  \item The adequacy of this encoding depends very strongly on its
    \emph{use}: this only truly encodes contexts if in every case an LF-variable
    is used for $X$.  Moreover, the various well-formedness constraints are a
    bit inelegant, as is the concept of mapping LF variables to De Bruijn indices.

  \item It is not possible to construct such a context on its own as a
    syntactic object of (LF-)type $\sortctx$, where variables bound earlier in
    the context may appear free later in the context. This can be done
    otherwise by constructing an object of type $\sortctx$ in an extended
    LF-context; the well-formedness of such a context is still an extrinsic
    property, though.

  \item A straightforward higher-order encoding is preferable (if possible)
    when working in the LF.

\end{enumerate}

\section{Encoding Telescopes in Higher-Order Abstract Syntax}
More precisely, it seems that what we really need is an encoding of
\emph{telescopes}, in the sense of De
Bruijn\cite{Bruijn91telescopicmappings}. A telescope is a sequence of
type assignments, where variables bound earlier may appear free later;
for instance:
\[
  \cdot,x_1:A_1, x_2:A_2(x_1), x_3:A_3(x_1,x_2), \dots, x_n:A_n(x_1,\dots,x_{n-1})
\]

To encode this in a straightforward, higher-order manner, we shall first
stipulate that each variable in the telescope will be an LF-variable. Then,
each type $A_i$ in the telescope will be a term with $i-1$ LF-bindings (i.e.\ it
will be an $(i-1)$-ary $\lambda$-abstraction). We'll define by mutual induction a
sort for telescopes $\sorttele$, and a sort for open terms with respect to a
telescope $\sortoexp{\Psi}$:

\begin{gather*}
  \infer{\sorttele:\type}{}
  \quad
  \infer{
    \sortoexp{\Psi}:\type
  }{
    \Psi:\sorttele
  }\\[6pt]
  \infer{\cdot:\sorttele}{}
  \quad
  \infer{
    \Psi,A:\sorttele
  }{
    \Psi:\sorttele &
    A:\sortoexp{\Psi}
  }\\[6pt]
  \infer{
    \nobind{M}:\sortoexp{\cdot}
  }{
    M:\sortexp
  }
  \quad
  \infer{
    \bind{x}{M(x)}:\sortoexp{\Psi,A}
  }{
    M:\sortexp\to\sortoexp\Psi &
    A:\sortoexp\Psi
  }
\end{gather*}

Note that we do not put any well-typedness constraints on the arguments of the
bound term constructors; we wish to separate object syntax from object
judgements; this is particularly important for an object theory which does not
have decidable type checking.

Closed expressions may be weakened by extraneous variables:
\begin{gather*}
  \infer{
    \weaken{\Psi}{M}{M'}:\type
  }{
    \Psi:\sorttele &
    M:\sortexp &
    M':\sortoexp\Psi
  }\\[6pt]
  \infer{
    \weaken{\cdot}{M}{\nobind{M}}
  }{
  }\quad
  \infer{
    \weaken{\Psi,A}{M}{\bind{x}{M'}}
  }{
    \weaken{\Psi}{M}{M'}
  }
\end{gather*}

A more general definition of weakening can be given, but the simple one above
suffices for our use-case.

\section{Encoding Functional Hypothetico-General Judgement}
Henceforth I shall use the term ``functional sequent judgement'' to mean a
hypothetico-general judgement which also requires functionality. First, before
we get to the full sequent judgement, we will need the concept of a
``general judgement''; this is a judgement which has variables from
a telescope free in it.

\subsection{Codes for General Judgements}
We will define \emph{codes} for such judgements and then
interpret them into proper object judgements (LF-types) later.

\begin{gather*}
  \infer{
    \sortoj\Psi:\type
  }{
    \Psi:\sorttele
  }\\[6pt]
  \infer{
    \qisset{A}:\sortoj\Psi
  }{
    A:\sortoexp\Psi
  }
  \quad
  \infer{
    \qeqset{A}{B}:\sortoj\Psi
  }{
    A:\sortoexp\Psi &
    B:\sortoexp\Psi
  }\\[6pt]
  \infer{
    \qmem{M}{A}:\sortoj\Psi
  }{
    M:\sortoexp\Psi &
    A:\sortoexp\Psi
  }
  \quad
  \infer{
    \qeqmem{M}{N}{A}:\sortoj\Psi
  }{
    M:\sortoexp\Psi &
    N:\sortoexp\Psi &
    A:\sortoexp\Psi
  }
\end{gather*}

\subsection{Extensionality}

The next thing to do is to specify how each of these judgement(-codes) behaves
operationally (i.e. by substituting its outermost variable); we will do so by
extending the LF signature with a new algorithmic judgement
$\sploot{\dname{J}}{\dname{J}'}$ and making it evident on all inputs:

\begin{gather*}
  \infer{
    \sploot{\dname{J}}{[x]\dname{J}'(x)}:\type
  }{
    \dname{J}:\sortoj{\Psi,A} &
    \dname{J}': \sortexp\to\sortoj\Psi
  }\\[6pt]
  \infer{
    \sploot{
      \qisset{\bind{x}{A(x)}}
    }{
      [x]\; \qisset{A(x)}
    }
  }{
  }\\[6pt]
  \infer{
    \sploot{
      \qeqset{\bind{x}{A(x)}}{\bind{x}{B(x)}}
    }{
      [x]\; \qeqset{A(x)}{B(x)}
    }
  }{
  }\\[6pt]
  \infer{
    \sploot{
      \qmem{\bind{x}{M(x)}}{\bind{x}{A(x)}}
    }{
      [x]\; \qmem{M(x)}{A(x)}
    }
  }{
  }\\[6pt]
  \infer{
    \sploot{
      \qeqmem{\bind{x}{M(x)}}{\bind{x}{N(x)}}{\bind{x}{A(x)}}
    }{
      [x]\; \qeqmem{M(x)}{N(x)}{A(x)}
    }
  }{
  }
\end{gather*}

Next, we will do something similar in order to show how each of these
judgement(-codes) behaves extensionally, by introducing another algorithmic
judgement $\refract{\dname{J}}{\dname{J}^=}$. Intuitively, this judgement
describes how what happens when you substitute two purportedly equal
expressions in for the judgement-code's outermost variable.

\begin{gather*}
  \infer{
    \refract{\dname{J}}{[x][y]\dname{J}'(x)(y)}:\type
  }{
    \dname{J}:\sortoj{\Psi,A} &
    \dname{J}': \sortexp\to\sortexp\to\sortoj\Psi
  }\\[6pt]
  \infer{
    \refract{
      \qisset{\bind{x}{A(x)}}
    }{
      [x][y]\; \qeqset{A(x)}{A(y)}
    }
  }{
  }\\[6pt]
  \infer{
    \refract{
      \qeqset{\bind{x}{A(x)}}{\bind{x}{B(x)}}
    }{
      [x][y]\; \qeqset{A(x)}{B(y)}
    }
  }{
  }\\[6pt]
  \infer{
    \refract{
      \qmem{\bind{x}{M(x)}}{\bind{x}{A(x)}}
    }{
      [x][y]\; \qeqmem{M(x)}{M(y)}{A(x)}
    }
  }{
  }\\[6pt]
  \infer{
    \refract{
      \qeqmem{\bind{x}{M(x)}}{\bind{x}{N(x)}}{\bind{x}{A(x)}}
    }{
      [x][y]\; \qeqmem{M(x)}{N(y)}{A(x)}
    }
  }{
  }
\end{gather*}

The judgement $\sploot{\dname{J}}{\dname{J}'}$ may be considered as
defining the behavior of $\dname{J}$ with one hole punched in it;
$[x]\;\dname{J}(x)$ could be called the ``one-hole application of
$\dname{J}$''. Likewise, $\refract{\dname{J}}{\dname{J}^=}$ may be
thought of as defining the behavior $\dname{J}$ with two hole punched
in it, and $[x][y]\; \dname{J}^=(x,y)$ could be called the ``two-hole
application of $\dname{J}$''.

\subsection{The Functional Sequent Judgement}

We finally have built up enough machinery to give an immediate and elegant
encoding to Martin-L\"of's functional sequent judgement $\dname{J}\ (\Psi)$;
we will notate it $\sequent{\Psi}{\dname{J}}$, or $\nece{\dname{J}}$ when
$\Psi$ is the empty telescope. First we will define it, and then we'll go back
and show some corresponding examples between our system and the informal rules
of MLTT~1979.

\begin{gather*}
  \infer{
    \sequent{\Psi}{\dname{J}}:\type
  }{
    \Psi:\sorttele &
    \dname{J}:\sortoj\Psi
  }
  \quad
  \infer{
    \nece{\dname{J}} \equiv \sequent{\cdot}{\dname{J}} : \type
  }{
  }
\end{gather*}

The interpretations of the judgement codes in the empty context are trivial:

\begin{gather*}
  \infer=[\sequentruleset]{
    \nece{\qisset{\nobind{A}}}
  }{
    \isset{A}
  }
  \quad
  \infer=[\sequentruleeqset]{
    \nece{\qeqset{\nobind{A}}{\nobind{B}}}
  }{
    \eqset{A}{B}
  }\\[6pt]
  \infer=[\sequentrulemem]{
    \nece{\qmem{\nobind{M}}{\nobind{A}}}
  }{
    \mem{M}{A}
  }
  \quad
  \infer=[\sequentruleeqmem]{
    \nece{\qeqmem{\nobind{M}}{\nobind{N}}{\nobind{A}}}
  }{
    \eqmem{M}{N}{A}
  }
\end{gather*}

The interpretations in open contexts may be done in one fell swoop using the
machinery that we have built up so far:

\begin{gather*}
  \infer{
    \sequent{\Psi,A}{\dname{J}}
  }{
    \begin{array}{c}
      \sploot{\dname{J}}{[x]\dname{J}'(x)}\\
      \refract{\dname{J}}{[x][y]\dname{J}^=(x)(y)}\\
      \lfpi{x,x'}{
        \weaken{\Psi}{x}{x'}
        \to\sequent{\Psi}{\qmem{x'}{A}}
        \to\sequent{\Psi}{\dname{J}'(x)}
      }\\
      \lfpi{x,x',y,y'}{
        \weaken{\Psi}{x}{x'}
        \to\weaken{\Psi}{y}{y'}
        \to\sequent{\Psi}{\qeqmem{x'}{y'}{A}}
        \to\sequent{\Psi}{\dname{J}^=(x)(y)}
      }\\
    \end{array}
  }
\end{gather*}

\section{The Definitions of the Types}
It is now possible to give the definitions of the types; recall that we will
only be causing judgements of the following forms to become evident:
$\canset{A}$, $\canmem{M}{A}$, $\eqcanmem{M}{N}{A}$; all other forms of
judgement have been fully explained. The ``spirit'' of Martin-L\"of's type
theory lies in the concept of evaluation of closed terms to canonical form, and
as such, defining a type means only describing how its canonical verifications
behave.

Let us recall Martin-L\"of's informal rules and we will show how they are
encoded in the logical framework. First, the cartesian product of a family of
sets is defined by the following rules in \cite{Martin-Lof-1979}:
\begin{gather*}
  \infer{
    \isset{(\mathsf{\Pi}x\in A)B}
  }{
    \isset{A} &
    \isset{B(x)}\;(\mem{x}{A})
  }\quad
  \infer{
    \mem{(\lambda x)b}{(\mathsf{\Pi}x\in A)B}
  }{
    \mem{b}{B}\ (\mem{x}{A})
  }\quad
  \infer{
    \eqmem{(\lambda x)b}{(\lambda x)d}{(\mathsf{\Pi}x\in A)B}
  }{
    \eqmem{b}{d}{B}\ (x\in A)
  }
\end{gather*}
%
These rules are adequately encoded in our system as follows:
\begin{gather*}
  \infer{
    \canset{\typi(A;B)}
  }{
    \isset{A} &
    \sequent{\cdot,A}{\qisset{\tacitbind{B}}}
  }\quad
  \infer{
    \canmem{\lam E}{\typi(A;B)}
  }{
    \sequent{\cdot,A}{\qmem{\tacitbind{E}}{\tacitbind{B}}}
  }\quad
  \infer{
    \eqcanmem{\lam E}{\lam E'}{\typi(A;B)}
  }{
    \sequent{\cdot,A}{\eqcanmem{\tacitbind{E}}{\tacitbind{E'}}{\tacitbind{B}}}
  }
\end{gather*}

The disjoint union of a family of sets is also easily encoded; first
Martin-L\"of's rules:
\begin{gather*}
  \infer{
    \isset{(\mathsf{\Sigma}x\in A)B}
  }{
    \isset{A} &
    \isset{B(x)}\;(\mem{x}{A})
  }\quad
  \infer{
    \mem{(a,b)}{(\mathsf{\Sigma}x\in A)B}
  }{
    \mem{a}{A} &
    \mem{b}{[a/x]B}
  }\quad
  \infer{
    \eqmem{(a,b)}{(c,d)}{(\mathsf{\Sigma}x\in A)B}
  }{
    \eqmem{a}{c}{A} &
    \eqmem{b}{d}{[a/x]B}
  }
\end{gather*}
%
and their encoding:
\begin{gather*}
  \infer{
    \canset{\tysg(A;B)}
  }{
    \isset{A} &
    \sequent{\cdot,A}{\qisset{\tacitbind{B}}}
  }\quad
  \infer{
    \canmem{\pair(M,N)}{\tysg(A;B)}
  }{
    \mem{M}{A} &
    \mem{N}{B(M)}
  }\quad
  \infer{
    \eqcanmem{\pair(M;N)}{\pair(M';N')}{\typi(A;B)}
  }{
    \eqmem{M}{M'}{A} &
    \eqmem{N}{N'}{B(M)}
  }
\end{gather*}

And what of the elimination rules? The entire point of the verificationist
meaning explanations which pervade \cite{Martin-Lof-1979} is that the
elimination rules will not be part of the definitions of the types, but rather
rules which are admissible, justified by the meanings of the judgements and the
definitions of the types. So, only after having propounded the definitions
above may we entertain \emph{justifying} the elimination rules in the logical
framework.

\subsection{Validity of Elimination Rules}

Here, we demonstrate the validity of the application rule for $\typi$
sets; the validity of further elimination rules is left for future
work.

\begin{theorem}
The standard application rule for the cartesian product of a family of
sets is admissible:
\[
  \infer[\typi_E]{
    \mem{\ap(M;N)}{B(N)}
  }{
    \dlabel{\dname D}{
      \mem{M}{\thunk{\typi(A;B)}}
    } &
   \dlabel{\dname E}{
      \mem{N}{A}
    }
  }
\]
\end{theorem}

\begin{proof}
  By induction on the derivation $\dname{D}$, which must have the following canonical form:
  \[
     \infer[\dname{D}]{
       \mem{M}{\thunk{\typi(A;B)}}
     }{
       \dlabel{\dname D}{\eval{M}{\lam E}} &
       \infer{
         \canmem{\lam E}{\typi(A;B)}
       }{
         \infer{
           \sequent{\cdot,A}{\qmem{\tacitbind{E}}{\tacitbind{B}}}
         }{
           \dlabel{\dname D''}{
             \lfpi{x,x'}{
               \weaken{\cdot}{x}{x'}
               \to\sequent{\cdot}{\qmem{x'}{A}}
               \to\sequent{\cdot}{\qmem{\nobind{E(x)}}{\nobind{B(x)}}}
             }
           } &
           \dlabel{}{\dots}
         }
       }
     }
  \]

  Admissibility follows from $\dname{D}'$, $\dname{D}''$, $\dname{E}$;
  let $\dname{F} \equiv
  \dname{D}''(N,\nobind{N},\weakenrulenil,\sequentrulemem\dname{E})$, which
  must be a canonical derivation as follows:
  \[
    \infer[\dname{F}]{
      \sequent{\cdot}{\qmem{\nobind{E(N)}}{\nobind{B(N)}}}
    }{
      \infer{
        \mem{E(N)}{B(N)}
      }{
        \dlabel{\dname F'}{\eval{E(N)}{X}} &
        \dlabel{\dname F''}{\eval{B(N)}{C}} &
        \dlabel{\dname F'''}{\canmem{X}{C}}
      }
    }
  \]
  Whence we may construct the goal:
  \medskip

  \framebox{
    \begin{minipage}{0.9\textwidth}
      \[
        \infer{
          \mem{\ap(M;N)}{B(N)}
        }{
          \infer{
            \eval{\ap(M;N)}{X}
          }{
            \dlabel{\dname D'}{\eval{M}{\lam E}} &
            \dlabel{\dname F'}{\eval{E(N)}{X}}
          }&
          \dlabel{\dname F''}{\eval{B(N)}{C}} &
          \dlabel{\dname F'''}{\canmem{X}{C}}
        }
      \]
    \end{minipage}
  }
\end{proof}


\section{Conclusion}

A rich, higher-order encoding for telescopes addresses several issues
which come up when encoding Martin-L\"of's extensional type theory
into the LF.

First, functionality premises no longer need be directly included in
every rule which uses hypothetico-general judgement. These are
factored out into a separate judgement which has a compositional
definition (the functional sequent judgement). Because of this, the
rules in the LF will look very much like how they did in
Martin-L\"of's informal presentation.

Secondly, the core judgements of type theory (i.e.\ typehood, equality
of types, membership, and equality of members) are defined first in
their atomic/categorical form, and then separately they are explained
with respect to a telescope, both directly and extensionally. Unlike
in Martin-L\"of's presentation, the sequent judgement is, then, not
defined separately at each categorical judgement, but is rather
explained once and for all in terms of the novel
$\sploot{\dname{J}}{\dname{J}'}$ and
$\refract{\dname{J}}{\dname{J}^=}$ judgements.

\bigskip

\bibliographystyle{acmtrans}
\nocite{*}
\bibliography{paper}

\end{document}

