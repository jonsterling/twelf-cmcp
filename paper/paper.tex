\documentclass[acmtoplas]{acmtrans2m}

\usepackage{amsmath}
\usepackage{amssymb}
\usepackage{booktabs}
\usepackage{proof}
\usepackage{twelf}
\usepackage[usenames,dvipsnames,svgnames,table]{xcolor}
\usepackage{code}

\newtheorem{theorem}{Theorem}[section]
\newtheorem{conjecture}[theorem]{Conjecture}
\newtheorem{corollary}[theorem]{Corollary}
\newtheorem{proposition}[theorem]{Proposition}
\newtheorem{lemma}[theorem]{Lemma}
\newdef{definition}[theorem]{Definition}
\newdef{remark}[theorem]{Remark}
\def\permission{}

\def\InputModeColorName{MidnightBlue}
\def\OutputModeColorName{Maroon}
\newcommand\InputMode[1]{{\color{\InputModeColorName}{#1}}}
\newcommand\OutputMode[1]{{\color{\OutputModeColorName}{#1}}}

\newcommand\naturals{\mathbb{N}}
\newcommand\type{\mathbf{type}}
\newcommand\tyvoid{\mathsf{void}}
\newcommand\tyunit{\mathsf{unit}}
\newcommand\ax{\mathsf{ax}}
\newcommand\ap{\mathsf{ap}}
\newcommand\sortval{\mathsf{val}}
\newcommand\sortexp{\mathsf{exp}}
\newcommand\sortctx{\mathsf{ctx}}
\newcommand\sorttele{\mathsf{tele}}
\newcommand\sortoexp[1]{[#1]\sortexp}
\newcommand\eval[2]{\InputMode{#1}\mathbin{\Rightarrow}\OutputMode{#2}}
\newcommand\thunk[1]{{\color{Gray}\uparrow}#1}
\newcommand\evalrule[1]{\Rightarrow_{#1}}
\newcommand\nobind[1]{{\color{Gray}!}#1}
\newcommand\bind[2]{{\color{Gray}@}[#1]#2}

\newcommand\isset[1]{\InputMode{#1}\;\mathit{set}}
\newcommand\eqset[2]{\InputMode{#1}=\InputMode{#2}\;\mathit{set}}
\newcommand\mem[2]{\InputMode{#1}\in\InputMode{#2}}
\newcommand\eqmem[3]{\InputMode{#1}=\InputMode{#2}\in\InputMode{#3}}

\newcommand\canset[1]{\InputMode{#1}\;\mathit{set}_\mathsf{v}}
\newcommand\eqcanset[2]{\InputMode{#1}=\InputMode{#2}\;\mathit{set}_\mathsf{v}}
\newcommand\canmem[2]{\InputMode{#1}\in_\mathsf{v}\InputMode{#2}}
\newcommand\eqcanmem[3]{\InputMode{#1}=\InputMode{#2}\in_\mathsf{v}\InputMode{#3}}
\newcommand\lfpi[2]{{\color{Gray}\forall\{#1\}}.\;#2}

\DeclareMathOperator{\typi}{\mathsf{pi}}
\DeclareMathOperator{\tysg}{\mathsf{sg}}
\DeclareMathOperator{\lam}{\mathsf{\lambda}}
\DeclareMathOperator{\pair}{\mathsf{pair}}
\DeclareMathOperator{\fst}{\mathsf{fst}}
\DeclareMathOperator{\snd}{\mathsf{snd}}

%% Krary Stuff
\DeclareMathOperator\isvarop{\mathsf{isvar}}
\DeclareMathOperator\boundedop{\mathsf{bounded}}
\DeclareMathOperator\orderedop{\mathsf{ordered}}
\DeclareMathOperator\precedesop{\mathsf{precedes}}

\newcommand\isvar[2]{\isvarop(\InputMode{#1};\OutputMode{#2})}
\newcommand\bounded[2]{\boundedop(\InputMode{#1};\InputMode{#2})}
\newcommand\ordered[1]{\orderedop(\InputMode{#1})}
\newcommand\precedes[2]{\precedesop(\InputMode{#1};\InputMode{#2})}
\newcommand\boundedrule[1]{\mathsf{bounded}_{#1}}
\newcommand\orderedrule[1]{\mathsf{ordered}_{#1}}
\newcommand\lessthan[2]{\InputMode{#1}<\InputMode{#2}}

\title{A Higher-Order Approach to Explicit Contexts in LF}

\author{JONATHAN M.\ STERLING}

\begin{abstract}
  There are two standard techniques for representing dependently typed calculi
  in the LF. The first is to re-use the LF contexts for the object contexts,
  which can lead to difficulties when the meaning of hypothetico-general
  judgement in the object language is stronger than in the LF; this is, for
  instance, the case in MLTT~1979, where non-trivial functionality obligations
  are incurred in the sequent judgement.

  Another technique advanced by Krary is to represent contexts explicitly, and
  then define a sequent judgement over them; this can be used to resolve the
  problem described above (and several others), but it comes at the cost of
  verbosity, and cannot be to encode \emph{telescopes}, which are contexts
  where each successive term has the previous bound variables free in it.  In
  this paper, I demonstrate a higher-order encoding of contexts and telescopes
  which can be used to faithfully encode the sequent judgement for MLTT~1979.
\end{abstract}

\category{}{}{}
\terms{Logical Frameworks}
\keywords{Contexts, telescopes}

\begin{document}

\maketitle

\section{Introduction}

To motivate this work, let us consider an encoding of MLTT~1979 in the logical
framework. The syntax contains two sorts, $\sortval$ (for canonical forms) and
$\sortexp$ (for non-canonical forms), and is defined in the following LF
signature: \footnote{LF signatures are presented in the style of inference
  rules for clarity. Object judgement forms have a \emph{mode} which expresses
  which arguments represent inputs, and which represent outputs; in our
  presentation, we color the inputs \emph{\InputMode\InputModeColorName}, and
  the outputs \emph{\OutputMode\OutputModeColorName}. Proper assignment of
modes may be used to construe a judgement as having algorithmic content.}

\begin{gather*}
  \infer{\sortval:\type}{}
  \quad
  \infer{\sortexp:\type}{}\\[6pt]
  %
  \infer{\tyvoid:\sortval}{}
  \quad
  \infer{\tyunit:\sortval}{}
  \quad
  \infer{
    \typi(A;[x]B\,x):\sortval
  }{
    A:\sortexp &
    B:\sortexp\to\sortexp
  }
  \quad
  \infer{
    \tysg(A;[x]B\,x):\sortval
  }{
    A:\sortexp &
    B:\sortexp\to\sortexp
  }\\[6pt]
  %
  \infer{\ax:\sortval}{}
  \quad
  \infer{
    \lam[x]E\,x:\sortval
  }{
    E:\sortexp\to\sortexp
  }
  \quad
  \infer{
    \pair(M;N):\sortval
  }{
    M:\sortexp &
    N:\sortexp
  } \\[6pt]
  %
  \infer{
    \thunk{M}:\sortexp
  }{
    M:\sortval
  }
  \quad
  \infer{
    \ap(M;N):\sortexp
  }{
    M:\sortexp &
    N:\sortexp
  }
  \quad
  \infer{
    \fst(M):\sortexp
  }{
    M:\sortexp
  }
  \quad
  \infer{
    \snd(M):\sortexp
  }{
    M:\sortexp
  }
\end{gather*}


Then, the language is accorded a simple big-step operational semantics via the
\framebox{$\eval{M}{N}$} judgement:

\begin{gather*}
  \infer{
    \eval{M}{N}:\type
  }{
    M:\sortexp &
    N:\sortval
  }\\[6pt]
  %
  \infer[\evalrule\uparrow]{
    \eval{\thunk{M}}{M}
  }{
  }
  \quad
  \infer[\evalrule\ap]{
    \eval{\ap(\thunk{\lam[x]E(x)};M)}{N}
  }{
    \eval{E\,M}{N}
  }\\[6pt]
  %
  \infer[\evalrule\fst]{
    \eval{\fst(\thunk{\pair(M;N)})}{M'}
  }{
    \eval{M}{M'}
  }
  \quad
  \infer[\evalrule\snd]{
    \eval{\fst(\thunk{\pair(M;N)})}{N'}
  }{
    \eval{N}{N'}
  }
\end{gather*}

\section{The Na\"ive Higher-Order Approach}

Informally, there are four basic categorical judgements in MLTT~1979:\\

\begin{tabular}{ll}
  \toprule
  Form & Pronunciation\\
  \midrule
  $\isset{A}$ & $A$ is a set\\
  $\eqset{A}{B}$ & $A$ and $B$ are equal sets\\
  $\mem{M}{A}$ & $M$ is a member of $A$\\
  $\eqmem{M}{N}{A}$ & $M$ and $N$ are equal members of $A$\\
  \bottomrule
\end{tabular}\\

\medskip\noindent The subjects of these judgements are terms of sort
$\sortexp$. However, implicit in Martin-L\"of's presentation are four prior
judgements which operate only on canonical forms (i.e. terms of sort
$\sortval$). We will therefore introduce the following LF signature to account
for this:

\begin{gather*}
  \infer{
    \canset{A}:\type
  }{
    A:\sortval
  }\quad
  \infer{
    \eqcanset{A}{B}:\type
  }{
    A:\sortval &
    B:\sortval
  }\quad
  \infer{
    \canmem{M}{A}:\type
  }{
    M:\sortval &
    A:\sortval
  }\quad
  \infer{
    \eqcanmem{M}{N}{A}:\type
  }{
    M:\sortval &
    N:\sortval &
    A:\sortval
  }\\[6pt]
  \infer{
    \isset{A}:\type
  }{
    A:\sortval
  }\quad
  \infer{
    \eqset{A}{B}:\type
  }{
    A:\sortexp &
    B:\sortexp
  }\quad
  \infer{
    \mem{M}{A}:\type
  }{
    M:\sortexp &
    A:\sortexp
  }\quad
  \infer{
    \eqmem{M}{N}{A}:\type
  }{
    M:\sortexp &
    N:\sortexp &
    A:\sortexp
  }
\end{gather*}

Then, we will give the meanings of the judgements which deal with non-canonical
forms first, anticipating the explanations of the canonical forms judgements.

\begin{gather*}
  \infer{
    \isset{A}
  }{
    \eval{A}{A'} &
    \canset{A'}
  }\quad
  \infer{
    \eqset{A}{B}
  }{
    \begin{array}{c}
      \eval{A}{A'} \\
      \eval{B}{B'}
    \end{array} &
      \eqcanset{A'}{B'}
  }\\[6pt]
  \infer{
    \mem{M}{A}
  }{
    \begin{array}{c}
      \eval{M}{M'}\\
      \eval{A}{A'}
    \end{array} &
    \canmem{M'}{A'}
  }\quad
  \infer{
    \eqmem{M}{N}{A}
  }{
    \begin{array}{c}
      \eval{M}{M'} \\
      \eval{N}{N'} \\
      \eval{A}{A'}
    \end{array} &
    \eqcanmem{M'}{N'}{A'}
  }
\end{gather*}

Following Martin-L\"of, we give an extensional equality for all types (i.e.\
two types are equal when they have the same membership and equivalence
relations); we could choose an intensional/structural equality for types, but
for our purposes here, it is not necessary.

\begin{gather*}
  \infer{
    \eqcanset{A}{B}
  }{
    \begin{array}{l}
      \lfpi{M}{\canmem{M}{A}\to\canmem{M}{B}}\\
      \lfpi{M}{\canmem{M}{B}\to\canmem{M}{A}}\\
      \lfpi{M,N}{\eqcanmem{M}{N}{A}\to\eqcanmem{M}{N}{B}}\\
      \lfpi{M,N}{\eqcanmem{M}{N}{B}\to\eqcanmem{M}{N}{A}}
    \end{array}
  }
\end{gather*}

Then, all we have to do is define the remaining canonical forms judgements with
respect to our syntax. For the base types, this is trivial:

\begin{gather*}
  \infer{
    \canset\tyvoid
  }{
  }\quad
  \infer{
    \canset\tyunit
  }{
  }\quad
  \infer{
    \canmem\ax\tyunit
  }{
  }\quad
  \infer{
    \eqcanmem\ax\ax\tyunit
  }{
  }
\end{gather*}

But for terms which involve binding, the rules will become more complicated.
Informally, Martin-L\"of gives the following formation rule for $\mathsf\Pi$
using hypothetico-general judgement:
\[
  \infer[*]{
    \canset{\mathsf{\Pi}(x\in A)B(x)}
  }{
    \isset{A} &
    \isset{B(x)}\;(\mem{x}{A})
  }
\]
So we might be tempted to do something similar in the LF, as follows:
\[
  \infer[*']{
    \canset{\typi(A;[x]B\,x)}
  }{
    \isset{A} &
    \lfpi{x}\mem{x}{A}\to\isset{B(x)}
  }
\]

This encoding, however, is not adequate, since the meaning of
hypothetico-general judgement in Martin-L\"of's informal metalangauge is not
captured by the above attempt. Recall that hypothetico-general judgement in
type theory implicitly requires \emph{functionality} of type and term families;
therefore, in order for our definition to be correct, we would have to adjust
the rule to include functionality:
\[
  \infer{
    \canset{\typi(A;[x]B\,x)}
  }{
    \isset{A} &
    \begin{array}{l}
      \lfpi{x}\mem{x}{A}\to\isset{B(x)}\\
      \lfpi{x,y}\eqmem{x}{y}{A}\to\eqset{B(x)}{B(y)}\\
    \end{array}
  }
\]

We have at least got closer, but even this is not really satisfactory; we were
unable to factor out the functionality constraints, and must be careful to
place them in all the right places in each and every rule. Contrast this with
Martin-L\"of's definition of hypothetico-general judgement, which includes
functionality from the start, allowing the other judgements to use this ``off
the shelf''.

\section{Explicit Contexts \`a la Krary}
In order to factor out functionality constraints, it will be necessary to
consider the use of contexts in the encoding, and then recast the judgements
above as being sequents, i.e.\ judgements with respect to a context. Krary's
first-order encoding of explicit contexts in the LF is essentially the
following:

\begin{gather*}
  \infer{\sortctx:\type}{}
  \quad
  \infer{\cdot:\sortctx}{}
  \quad
  \infer{
    (\Gamma,X:A):\sortctx
  }{
    \Gamma:\sortctx &
    X:\sortexp &
    A:\sortexp
  }\\[6pt]
  \infer{
    \isvar{X}{I} : \type
  }{
    X:\sortexp &
    I:\naturals
  }
  \quad
  \infer{
    \precedes{X}{Y}: \type
  }{
    X:\sortexp &
    Y:\sortexp
  }\quad
  \infer{
    \precedes{X}{Y}
  }{
    \isvar{X}{I} &
    \isvar{Y}{J} &
    \lessthan{I}{J}
  }\\[6pt]
  \infer{
    \bounded{\Gamma}{X}:\type
  }{
    \Gamma:\sortctx &
    X:\sortexp
  }
  \quad
  \infer{
    \bounded{\cdot}{X}
  }{
    \isvar{X}{I}
  }
  \quad
  \infer{
    \bounded{\Gamma,Y:A}{X}
  }{
    \precedes{Y}{X} &
    \bounded{G}{Y}
  }
  \\[6pt]
  \infer{
    \ordered\Gamma:\type
  }{
    \Gamma:\sortctx
  }
  \quad
  \infer{\ordered\cdot}{}
  \quad
  \infer{
    \ordered{\Gamma,X:A}
  }{
    \bounded{\Gamma}{X}
  }
  \\[6pt]
\end{gather*}

Note that the well-formedness constraints above do not in practice protrude
into the definitions of judgements which \emph{use} explicit contexts; they are
only used in metatheorems. This encoding is very practical for certain calculi,
but it is concerning in our case for a few reasons:

\begin{enumerate}

  \item The correctness of this encoding depends very strongly on its
    \emph{use}: this only truly encodes contexts if in every case a LF-variable
    is used for $X$.  Moreover, the various well-formedness constraints are a
    bit inelegant, as is the concept of mapping LF variables to indices.

  \item It is not possible to construct such a context on its own as a
    syntactic object of (LF-)type $\sortctx$, where variables bound earlier in
    the context may appear free later in the context. This can be done
    otherwise by constructing an object of type $\sortctx$ in an extended
    LF-context; the well-formedness of such a context is still an extrinsic
    property, though.

  \item A straightforward higher-order encoding is preferable (if possible)
    when working in the LF.

\end{enumerate}

\section{Encoding Telescopes in Higher-Order Abstract Syntax}
More precisely, it seems that what we really need is an encoding of
\emph{telescopes}, in the sense of De Bruijn. A telescope is essentially a
sequence of type assignments, where variables bound earlier may appear free
later, for instance:
\[
  \cdot,x_1:A_1, x_2:A_2(x_1), x_3:A_3(x_1,x_2), \dots, x_n:A_n(x_1,\dots,x_{n-1})
\]

To encode this in a straightforward, higher-order manner, we shall first
stipulate that each variable in the telescope will be an LF-variable. Then,
each type $A_i$ in the telescope will be a term with $i-n$ bindings (i.e.\ it
will be an $i$-ary $\lambda$-abstraction). We'll define by mutual induction a
sort for telescopes $\sorttele$, and a sort for open terms with respect to a
telescope $\sortoexp{\Psi}$:

\begin{gather*}
  \infer{\sorttele:\type}{}
  \quad
  \infer{
    \sortoexp{\Psi}:\type
  }{
    \Psi:\sorttele
  }\\[6pt]
  \infer{\cdot:\sorttele}{}
  \quad
  \infer{
    \Psi,A:\sorttele
  }{
    \Psi:\sorttele &
    A:\sortoexp{\Psi}
  }\\[6pt]
  \infer{\nobind{M}:\sortoexp{\cdot}}{}
  \quad
  \infer{
    \bind{x}{M(x)}:\sortoexp{\Psi,A}
  }{
    M:\sortexp\to\sortoexp\Psi
  }
\end{gather*}

Note that we do not put any well-typedness constraints on the arguments of the
bound term constructors; we wish to separate object syntax from object
judgements; this is particularly important for an object theory which does not
have decidable type checking.

\end{document}

